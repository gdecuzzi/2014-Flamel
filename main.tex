\documentclass{article}
\usepackage[T1]{fontenc} %%%key to get copy and paste for the code!
\usepackage[utf8]{inputenc} %%% to support copy and paste with accents for frnehc stuff
\usepackage{times}
\usepackage[scaled=0.85]{helvet}
\usepackage{graphicx}
\usepackage{ifthen}
\usepackage{xspace}
\usepackage{alltt}
\usepackage{latexsym}
\usepackage{url}            
\usepackage{amssymb}
\usepackage{amsfonts}
\usepackage{amsmath}
\usepackage{stmaryrd}
\usepackage{enumerate}
\usepackage{cite}
\usepackage[pdftex,colorlinks=true,pdfstartview=FitV,linkcolor=blue,citecolor=blue,urlcolor=blue]{hyperref}
\usepackage{xspace}


\newboolean{showcomments}
\setboolean{showcomments}{true}
\ifthenelse{\boolean{showcomments}}
  {\newcommand{\bnote}[2]{
	\fbox{\bfseries\sffamily\scriptsize#1}
    {\sf\small$\blacktriangleright$\textit{#2}$\blacktriangleleft$}
    % \marginpar{\fbox{\bfseries\sffamily#1}}
   }
   \newcommand{\cvsversion}{\emph{\scriptsize$-$Id: macros.tex,v 1.1.1.1 2007/02/28 13:43:36 bergel Exp $-$}}
  }
  {\newcommand{\bnote}[2]{}
   \newcommand{\cvsversion}{}
  } 


\newcommand{\here}{\bnote{***}{CONTINUE HERE}}
\newcommand{\nb}[1]{\bnote{NB}{#1}}

\newcommand{\fix}[1]{\bnote{FIX}{#1}}
%%%% add your own macros 

\newcommand{\ab}[1]{\bnote{Alex}{#1}}
\newcommand{\sd}[1]{\bnote{Stef}{#1}}
\newcommand{\ja}[1]{\bnote{Jannik}{#1}}
\newcommand{\md}[1]{\bnote{MD}{#1}}
\newcommand{\jr}[1]{\bnote{JRe}{#1}}
\newcommand{\lf}[1]{\bnote{Luc}{#1}}

\graphicspath{{figures/}}
%%% 


\newcommand{\figref}[1]{Figure~\ref{fig:#1}}
\newcommand{\figlabel}[1]{\label{fig:#1}}
\newcommand{\tabref}[1]{Table~\ref{tab:#1}}
\newcommand{\layout}[1]{#1}
\newcommand{\commented}[1]{}
\newcommand{\secref}[1]{Section \ref{sec:#1}}
\newcommand{\seclabel}[1]{\label{sec:#1}}

%\newcommand{\ct}[1]{\textsf{#1}}
\newcommand{\stCode}[1]{\textsf{#1}}
\newcommand{\stMethod}[1]{\textsf{#1}}
\newcommand{\sep}{\texttt{>>}\xspace}
\newcommand{\stAssoc}{\texttt{->}\xspace}

\newcommand{\stBar}{$\mid$}
\newcommand{\stSelector}{$\gg$}
\newcommand{\ret}{\^{}}
\newcommand{\msup}{$>$}
%\newcommand{\ret}{$\uparrow$\xspace}

\newcommand{\myparagraph}[1]{\noindent\textbf{#1.}}
\newcommand{\eg}{\emph{e.g.,}\xspace}
\newcommand{\ie}{\emph{i.e.,}\xspace}
\newcommand{\etal}{\emph{et al.,}\xspace}
\newcommand{\ct}[1]{{\textsf{#1}}\xspace}


\newenvironment{code}
    {\begin{alltt}\sffamily}
    {\end{alltt}\normalsize}

\newcommand{\defaultScale}{0.55}
\newcommand{\pic}[3]{
   \begin{figure}[h]
   \begin{center}
   \includegraphics[scale=\defaultScale]{#1}
   \caption{#2}
   \label{#3}
   \end{center}
   \end{figure}
}

\newcommand{\twocolumnpic}[3]{
   \begin{figure*}[!ht]
   \begin{center}
   \includegraphics[scale=\defaultScale]{#1}
   \caption{#2}
   \label{#3}
   \end{center}
   \end{figure*}}

\newcommand{\infe}{$<$}
\newcommand{\supe}{$\rightarrow$\xspace}
\newcommand{\di}{$\gg$\xspace}
\newcommand{\adhoc}{\textit{ad-hoc}\xspace}

\usepackage{url}            
\makeatletter
\def\url@leostyle{%
  \@ifundefined{selectfont}{\def\UrlFont{\sf}}{\def\UrlFont{\small\sffamily}}}
\makeatother
% Now actually use the newly defined style.
\urlstyle{leo}







\begin{document}
\title{A cool title}
\maketitle


\title{Title that Describes the Contribution that Solves a Problem}
\author{O. Thor \and C. O. Ottohr}
\date{\today}
\maketitle

\begin{abstract}
In this context...
We consider this problem P...
P is a problem because...
We propose this solution...
Our solution solves P in such and such way.
\end{abstract}


\section{Introduction}
\label{sec:intro}

Context

Problem

- Problema: 
Quiero tener reglas de reescritura, para poder transformar código, 
por ejemplo para hacer refactors, para tener buscadores de más alto nivel, browsear código más fácil
simplificar todas esas tareas
dejar una API que permita hacerlo programáticamente

- Por qué es interesante: 

- Como lo atacamos: 
hay una herramienta desarrollada, tiene algunos problemas, hay un blog sobre el tema.

- En que se diferencia de cosas ya hechas: Al ser parte del lenguaje, liberar memoria pasa a ser un concern de alto nivel, que podríamos cambiar, reprogramar y mejorar. Otras soluciones que intentan mejorar el uso de memoria implican (a) el cambio de máquinas virtuales o la compactación de código (bytecode) dificultando el cambio (b) escribir implementaciones ad-hoc que pueden resolver los problemas solo parcialmente, en vez de tratarlo como un concern que atraviesa varias aplicaciones.


Known tracks for solutions
	here you want to show that you are not an idiot not knowing what have been around

What our solution is \ct{Set} and \ct{OrderedCollection} (so that the reader knows where the paper is going)

Contribution of the paper
una forma sencilla de definir búsquedas y transformaciones programáticamente
búsqueda y transformación basados en ejemplos (utilizando una herramienta visual para manipular los ejemplos

Paper structure




\section{Problem Description}
\label{sec:problem}

Context, exposed with the \textbf{most precise terms possible} (don't open
unwanted doors for the reader)


Probably set the vocabulary before to cut any misinterpretation

Constraints that influenced the solution (because the solution is not
universal) \emph{e.g.} our requirements for a solution, possibly not all
satisfied. They should be sound and believable. Analysis of the criteria.
Imagine that you are another guy having this problem do the constraint
matches yours so that you could apply the solution

Problem = no hay una forma fácil de hacer transformaciones y búsquedas genéricas orientadas a un ejémplo

Factual solution tracks, to position...
Our solution in a nutshell.


\section{Proposed Solution}
\label{sec:contribution}

Free form, variable number of sections, technical details.

But in general do not mix solution and discussions/possible variation
let that for discussion

\section{Discussion}
\label{sec:discussion}

Discussion of actual solution \emph{vs.} initial constraints from
\ref{sec:problem}. Explain the space of the solution, why we made it this way.

Evaluation of the solution. How does the solution meet the criteria? Where
does it succeed or fails...


\section{Related Works}
\label{sec:related}

Other solutions in the domain, and a real comparison of our contribution with
solutions from other people.


\section{Conclusion}
\label{sec:conclusion}

In this paper, we looked at problem P with this context and these
constraints. We proposed solution S. It has such good points and such not so
good ones. Now we could do this or that.


\section*{macro example}

\ct{look at it this is code }
\begin{code}{}
Class>>nknkjbkjbkjb
    | grgr | 
    grgrgrgg 
    a := 
\end{code}



\subsection*{Acknowledgements} This work was supported by Ministry of Higher Education and Research, Nord-Pas de Calais Regional Council, FEDER through the 'Contrat de
Projets Etat Region (CPER) 2007-2013',  the Cutter ANR project, ANR-10-BLAN-0219 and the MEALS Marie Curie Actions program FP7-PEOPLE-2011-
IRSES MEALS (no. 295261). 

% \bibliographystyle{plain}
% \bibliography{foo.bib}

% \appendix
% 
% \section{Lots of Furry Technical Details}

\bibliographystyle{abbrv}
\bibliography{scg}
\end{document}

%%% Local Variables: 
%%% coding: utf-8
%%% mode: latex
%%% TeX-master: "main"
%%% TeX-PDF-mode: t
%%% End:
